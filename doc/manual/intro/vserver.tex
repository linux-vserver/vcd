\chapter{The Linux-VServer Project}

The Linux-VServer project implements operating system-level virtualization
based on \textit{Security Contexts} which permit the creation of many
independent \textit{Virtual Private Servers} (VPS) that run simultaneously on a
single physical server at full speed, efficiently sharing hardware resources.

A VPS provides an almost identical operating environment as a conventional
Linux server. All services, such as ssh, mail, web and databases, can be
started on such a VPS, without (or in special cases with only minimal)
modification, just like on any real server.

Each virtual server has its own user account database and root password and is
isolated from other virtual servers, except for the fact that they share the
same hardware resources.


\section{Rationale}

Over the years, computers have become sufficiently powerful to use
virtualization to create the illusion of many smaller virtual machines, each
running a separate operating system instance.

There are several kinds of \textit{Virtual Machines} (VMs) providing similar
features, but differ in the degree of abstraction and the methods used for
virtualization.

Most of them accomplish what they do by emulating some real or fictional
hardware, which in turn consume real resources from the \textit{Host} (the
machine running the VMs). This approach, used by most system emulators, allows
the emulator to run an arbitrary operating systems, even for a different
hardware architecture. No modifications need to be made to the \textit{Guest}
(the operating system running in the VM) because it isn't aware of the fact
that it isn't running on real hardware.

Some system emulators require small modifications or specialized drivers to be
added to Host or Guest to improve performance and minimize the overhead
required for hardware emulation. Although this significantly improves
efficiency, there are still large amounts of resources being wasted in caches
and mediation between Guest and Host.

But suppose you do not want to run many different operating systems
simultaneously on a single box. Most applications running on a server do not
require hardware access or kernel level code, and could easily share a machine
with others, if they could be separated and secured...


\section{The Concept}

At a basic level, a Linux server consists of three building blocks: hardware,
kernel and applications. The hardware usually depends on the provider or system
maintainer, and, while it has a big influence on the overall performance, it
cannot be changed that easily, and will likely differ from one setup to
another.

The main purpose of the kernel is to build an abstraction layer on top of the
hardware to allow processes (applications) to work with and operate on
resources without knowing the details of the underlying hardware. Ideally,
those processes would be completely hardware agnostic, by being written in an
interpreted language and therefore not requiring any hardware-specific
knowledge.

Given that a system has enough resources to drive ten times the number of
applications a single Linux server would usually require, why not put ten
servers on that box, which will then share the available resources in an
efficient manner?

Most server applications (e.g. httpd) will assume that it is the only
application providing a particular service, and usually will also assume a
certain filesystem layout and environment. This dictates that similar or
identical services running on the same physical server, but for example, only
differing in their addresses, have to be coordinated. This typically requires a
great deal of administrative work which can lead to reduced system stability
and security.

The basic concept of the Linux-VServer solution is to separate the user-space
environment into distinct units (Virtual Private Servers) in such a way that
each VPS looks and feels like a real server to the processes contained within.

Although different Linux distributions use (sometimes heavily) patched kernels
to provide special support for unusual hardware or extra functionality, most
Linux distributions are not tied to a special kernel.

Linux-VServer uses this fact to allow several distributions, to be run
simultaneously on a single, shared kernel, without direct access to the
hardware, and share the resources in a very efficient way.


\section{Usage Scenarios}

The primary goal of this project is to create virtual servers sharing the same
machine. A virtual server operates like a normal Linux server. It runs normal
services such as telnet, mail servers, web servers, and database servers.


\subsection{Administrative Separation}

As the hardware evolves, it is tempting to put more and more tasks on a server.
Though Linux could reliably handle it, at some point, the system administrator
will likely end up with too much stuff and users on one system and worrying
about system updates. Additionally, separating different or similar services
which otherwise would interfere with each other, either because they are poorly
designed or because they are simply incapable of peaceful coexistence for
whatever reason, may often be complex or even impossible.

The Linux-VServer project addresses this issue. The same box is able to run
multiple virtual servers and each one does the job it is supposed to do. If the
administrator needs to upgrade to PHP 5 for a given project, he can update just
one virtual server and does not affect the others.

Also, the root password of a virtual servers can be given to foreign
administrators, thus allows him to perform updates, restart services or update
system configuration without having to know or worry about other virtual
servers hosted on the same machine. This allows a clever provider to sell
Virtual Private Servers, which uses less resources than other virtualization
techniques, which in turn allows to put more units on a single machine.

The list of providers doing so is relatively long, and so this is rightfully
considered the main area of application. See the Linux-VServer project
page\footnote{\url{http://linux-vserver.org/VServer_Hosting}} for a (possibly
incomplete) list of companies providing Virtual Private Servers based on the
Linux-VServer technology.


\subsection{Enhancing Security}

While it can be interesting to run several virtual servers in one box, there is
one concept potentially more generally useful. Imagine a physical server
running a single virtual server. The goal is to isolate the main environment
from any service, any network. You boot in the main environment, start very few
services and then continue in the virtual server.

The service in the main environment would be:

\begin{itemize}
	\item Unreachable from the network.

	\item Able to log messages from the virtual server in a secure way. The
		virtual server would be unable to change/erase the logs. Even a cracked
		virtual server would not be able the edit the log.

	\item Able to run intrusion detection facilities, potentially spying the
		state of the virtual server without being accessible or noticed. For
		example, tripwire could run there and it would be impossible to
		circumvent its operation or trick it.
\end{itemize}

Another option is to put the firewall in a virtual server, and pull in the DMZ,
containing each service in a separate VPS. On proper configuration, this setup
can reduce the number of required machines drastically, without impacting
performance.


\subsection{Resource Indepedence}

Since virtual servers are only guests on the hardware they are using, they are
not aware of the specifics: they do not contain disk configurations, kernels or
network configurations.

One key feature of a virtual server is the independence from the actual
hardware. Most hardware issues are irrelevant for a virtual server
installation.

The main server acts as a host and takes care of all the details. The virtual
server is just a client and ignores all the details. As such, the client can be
moved to another physical server with very few manipulations.

For example, to move the virtual server from one physical computer to another,
it sufficient to do the following:

\begin{itemize}
	\item shutdown the running virtual server
	\item copy it over to the other machine
	\item copy the configuration
	\item start the virtual server on the new machine
\end{itemize}

No adjustments to user setup, password database or hardware configuration are
required, as long as both machine architectures are binary compatible.

Thus, once a virtual server is using more resource than expected, the
administrator can easily move it to another machine without the need to worry
about configuration files for disk layout, network interfaces etc. A virtual
server is just a directory on the filesystem of host system.


\subsection{Distribution Independence}

People are often talking about their preferred distribution. Should one use
Fedora, Debian or something else? Should one give a spin to the latest and
greatest distribution just for the sake of it?

With virtual servers, the choice of a distribution is less important. When you
select a distribution, you expect it will do the following:

\begin{itemize}
	\item Good hardware support/detection
	\item Good package technology/updates
	\item Good package selection
	\item Reliable packages
\end{itemize}

The choice is important because every service running on a box will be using
the same distribution. Most distributions out there are good and reliable.
Still each one has its peculiarities and probably flaws. For example, one
distribution is doing a great job on security but is not delivering the latest
and greatest PHP. Now because you have decided to use this distribution for
some projects, using virtual servers does not prevent you from using another
distribution for other projects or even a second virtual server for existing
projects.


\subsection{Fail-over Scenarios}

Pushing the limit a little further, replication technology could be used to
keep an up-to-the-minute copy of the filesystem of a running virtual server.
This would permit a very fast fail-over if the running server goes offline for
whatever reason.

All the known methods to accomplish this, starting with network replication via
rsync, or drbd, via network devices, or shared disk arrays, to distributed
filesystems, can be utilized to reduce the down-time and improve overall
efficiency.


\subsection{Experimenting and Upgrading}

If the system administrator intends to upgrade a system to get new features or
security updates, he probably first wants to test new packages on a development
machine, before the production server can be updated. Under normal
circumstances, i.e. all test have been passed, the upgrade procedure will look
something like this:

\begin{itemize}
	\item Doing a backup of the server
	\item Perform all the upgrades and install the new applications
\end{itemize}

Two hours later something does not work as expected and -- to make it even
worse -- it works fine on the development machine. Every system administrator
has experienced this scenario at least once.

Another solution to this problem would be to install the new production server
on new hardware, but this is not always possible, due to lack of hardware or
the effort needed to clone an existing machine.

Using virtual servers, all this becomes very easy:

\begin{itemize}
	\item Stop the virtual server in production
	\item Make a copy of the virtual server
	\item Perform the upgrades in the new virtual server
\end{itemize}

To get back to the example from above, two hours later something does not work
as expected and there is no immediate fix for the problem.

Again, using virtual servers, the (temporary) solution to this problem is very
easy:

\begin{itemize}
	\item Stop the new virtual server and assign it a new IP address
	\item Start both the old and new virtual server
\end{itemize}

Now the old one is still online and the issue can be tracked down on your new
virtual server using a different IP address. After the problem has been fixed
the new virtual server will be reassign the old IP address and serve for
production.


% TODO: more info needed here
\subsection{Development and Testing}

Consider a software tool or package which should be built for several versions
of a specific distribution (Mandrake 8.2, 9.0, 9.1, 9.2, 10.0) or even for
different distributions.

This is easily solved with Linux-VServer. Given plenty of disk space, the
different distributions can be installed and running side by side, simplifying
the task of switching from one to another.

Of course this can be accomplished by chroot() alone, but with Linux-VServer
it's a much more realistic simulation.


% TODO: more info needed here
\section{History}

\textbf{Jacques Gélinas} created the VServer project a number of years back.
He still does vserver development and the community can be glad to have him.
He's a genius, without him, Linux-VServer would not exist. Three cheers for
Jack.

But sometime during 2003 it became apparent that Jack didn't have the time to
keep vserver development up to pace. So in November, \textbf{Herbert Pötzl}}
officially took charge of development. He now releases the vserver kernel
patches, announcing them on the vserver mailing list and making them available
for the public.

Additionally, \textbf{Enrico Scholz} decided to reimplement Jack's vserver
tools in \textbf{C}.  These are now distributed as util-vserver. They are
backward compatible to Jack's tools as far as possible, but follow the kernel
patch development more closely.

In 2005, \textbf{Benedikt Böhm} started another reimplementation of the
userspace utilities, although with a completely different architecture in mind.
This new implementation is known as VServer Control Daemon as you already may
know when you read this manual \texttt{;-)}
