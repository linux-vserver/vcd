\addchap{Preface}



\addsec{Audience}

This book is written for computer-literate folk who want to use the
Linux-VServer technology to seperate their processes (applications) into
distinct execution units for one or more of the following reasons:

\begin{enumerate}
\item Administrative Seperation
\item Service Seperation
\item Enhanced Security
\item Easy Maintenance
\item Fail-over Scenarios
\item Development and Testing
\end{enumerate}

Most readers are probably system administrators who need to seperate their
applications or programmers who want to test their programs in many different
distributions or configurations.

Although this book is written to cover the whole extend of the Linux-VServer
technology, it is advisable to have basic knowledge about the Linux operating
system, its shell and commands as well as your distribution of choice with all
its peculiarities.



\addsec{Organization of This Book}



\addsec{Conventions Used in This Book}

\begin{labeling}{\texttt{monospace}}
\item[\textit{italic}] An italic font is used for filenames, URLs, emphasized
text, and the first usage of technical terms.

\item[\texttt{monospace}] A monospaced font is used for error messages,
commands, environment variables, names of ports, hostnames, user names, group
names, device names, variables, and code fragments.

\item[\textbf{bold}] A bold font is used for applications, commands, and keys.
\end{labeling}



\addsec{Acknowledgements}



\addsec{Feedback}

If you found a typo in this manual, or if you have thought of a way to make this
guide better, feel free to contact the authors!

If you have suggestions for improving this manual, try to be as specific as
possible when formulating it. If you have found an error, please include the
chapter/section/subsection name and some of the surrounding text so we can fine
it easily.

Please submit a report by e-mail to one of the addresses mentioned above.
