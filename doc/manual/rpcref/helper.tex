\chapter{Helper Methods}


% helper.netup
\section{helper.netup}

This method is used to setup necessary network configuration during startup of
virtual servers. It is called by \texttt{vshelper} once the kernel has created
the network context. This method is not supposed to be called by a user or
administrator.

\begin{rpcsynopsis}{helper.netup}{int xid}
\rpcparam{xid}{Unique context ID}
\end{rpcsynopsis}

\begin{rpcaccess}
\rpccapability{HELPER} and \rpcnoownerchecks.
\end{rpcaccess}

\rpcreturnnil

\rpcnoerrors


% helper.restart
\section{helper.restart}

This method is used to remember reboot requests from vitrtual servers. It is
called by \texttt{vshelper} once virtual servers have issued a reboot system
call. After the request has been stored in VXDB the kernel terminates all
processes belonging to the virtual server, and calls \texttt{helper.shutdown}
afterwards.  This method is not supposed to be called by a user or
administrator.

\begin{rpcsynopsis}{helper.restart}{int xid}
\rpcparam{xid}{Unique context ID}
\end{rpcsynopsis}

\begin{rpcaccess}
\rpccapability{HELPER} and \rpcnoownerchecks.
\end{rpcaccess}

\rpcreturnnil

\rpcnoerrors


% helper.shutdown
\section{helper.shutdown}

This method is used to cleanup state data after virtual server shutdown. If a
reboot request has been recorded in VXDB by \texttt{helper.restart} the virtual
server is started again. This method is not supposed to be called by a user or
administrator.

\begin{rpcsynopsis}{helper.shutdown}{int xid}
\rpcparam{xid}{Unique context ID}
\end{rpcsynopsis}

\begin{rpcaccess}
\rpccapability{HELPER} and \rpcnoownerchecks.
\end{rpcaccess}

\rpcreturnnil

\rpcnoerrors



% helper.startup
\section{helper.startup}

This method is used to setup necessary configuration during startup. This
includes all configuration except network configuration, which is handled in
\texttt{helper.netup}. % FIXME: cross-ref to explanation of startup procedure

\begin{rpcsynopsis}{helper.startup}{int xid}
\rpcparam{xid}{Unique context ID}
\end{rpcsynopsis}

\begin{rpcaccess}
\rpccapability{HELPER} and \rpcnoownerchecks.
\end{rpcaccess}

\rpcreturnnil

\rpcnoerrors
