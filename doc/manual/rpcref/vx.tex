\chapter{General Maintenance Methods}

The vx family of functions provide general maintenance facilities for virtual
servers.


% vx.create
\section{vx.create}

% FIXME: add cross-ref to template explanation
This method is used to create a new virtual server using a template cache.

\begin{rpcsynopsis}{vx.create}{string name, string template,
	bool force, bool copy, string vdir}
\rpcparam{name}{Unique name to use for the new virtual server}
\rpcparam{template}{Name of the template cache to use for the new virtual
	server}
\rpcparam{rebuild}{If the virtual server already exists and this is set to
	true a recursive unlink is performed before the template extraction
	takes place}
\end{rpcsynopsis}

\begin{rpcaccess}
\rpccapability{CREATE} and \rpcownerchecks. The administrator passes owner
check even if the specified virtual server does not exist, in order to be able
to create new ones. This differs slightly from the normal owner check
behaviour.
\end{rpcaccess}

\rpcreturnnil

% FIXME: update errors
\begin{rpcerrors}
\rpcerror{MEINVAL}{An empty template cache name was supplied, or the supplied
	value contained invalid characters}
\rpcerror{MEEXIST}{The virtual server already exists and the rebuild parameter
	was not set to true}
\rpcerror{MERUNNING}{The virtual server already exists, the rebuild parameter
	was set to true, but the existing virtual server is still running}
\end{rpcerrors}


% vx.exec
\section{vx.exec}

This method is used to execute a simple command inside a virtual server. This
method does not support redirection, pipes and other shell stuff, the command
line is executed with a plain \texttt{execvp}.

\begin{rpcsynopsis}{vx.exec}{string name, string command}
\rpcparam{name}{Unique virtual server name}
\rpcparam{command}{Simple command line to execute}
\end{rpcsynopsis}

\begin{rpcaccess}
\rpccapability{EXEC} and \rpcownerchecks.
\end{rpcaccess}

\rpcreturnsimple{a combined string of \texttt{STDOUT} and \texttt{STDERR}}

\begin{rpcerrors}
\rpcerror{MESTOPPED}{The specified virtual server is currently not running,
	therefore the command cannot be executed}
\end{rpcerrors}


% vx.kill
\section{vx.kill}

This method is used to send signals to processes inside a virtual server.

\begin{rpcsynopsis}{vx.kill}{string name, int pid, int sig}
\rpcparam{name}{Unique virtual server name}
\rpcparam{pid}{Process ID to send signal to, there are two special cases:\\
	0 = send signal to all processes\\
	1 = send signal to the real pid of a faked init}
\rpcparam{sig}{Signal Number, see the \texttt{signal(7)} man page for more
	information}
\end{rpcsynopsis}

\begin{rpcaccess}
\rpccapability{INIT} and \rpcownerchecks.
\end{rpcaccess}

\rpcreturnnil

\begin{rpcerrors}
\rpcerror{MESTOPPED}{The specified virtual server is currently not running,
	therefore no signals can be sent}
\end{rpcerrors}


% vx.load
\section{vx.load}

This method is used to receive the current load average, i.e. the number of
jobs in the run queue (state R) or waiting for disk I/O (state D) averaged over
1, 5, and 15 minutes.

\begin{rpcsynopsis}{vx.load}{string name}
\rpcparam{name}{Unique virtual server name}
\end{rpcsynopsis}

\begin{rpcaccess}
\rpccapability{INFO} and \rpcownerchecks.
\end{rpcaccess}

\begin{rpcreturncomplex}{a \texttt{struct}}{string 1m, string 5m, string 15m}
\rpcreturnparam{1m}{Load average in the last minute}
\rpcreturnparam{5m}{Load average in the last 5 minutes}
\rpcreturnparam{15m}{Load average in the last 15 minutes}
\end{rpcreturncomplex}

\rpcnoerrors


% vx.reboot
\section{vx.reboot}

% FIXME: add cross-ref to reboot explanation
This method is used as a wrapper to vx.exec using the configured
\texttt{reboot} command.

\begin{rpcsynopsis}{vx.reboot}{string name}
\rpcparam{name}{Unique virtual server name}
\end{rpcsynopsis}

\begin{rpcaccess}
\rpccapability{INIT} and \rpcownerchecks.
\end{rpcaccess}

\rpcreturnsimple{the return value of \texttt{vx.exec}}

\begin{rpcerrors}
\rpcerror{MESTOPPED}{The specified virtual server is currently not running,
	therefore it cannot be stopped}
\end{rpcerrors}


% vx.remove
\section{vx.remove}

This method is used to remove a virtual server from the configuration database
and filesystem.

\begin{rpcsynopsis}{vx.remove}{string name}
\rpcparam{name}{Unique virtual server name}
\end{rpcsynopsis}

\begin{rpcaccess}
\rpccapability{CREATE} and \rpcownerchecks.
\end{rpcaccess}

\rpcreturnnil

\begin{rpcerrors}
\rpcerror{MERUNNING}{The specified virtual server is currently running,
	therefore it cannot be removed}
\rpcerror{MEEXIST}{The root filesystem fo the virtual server still exists after
	removal and no mount point exists for it. Manual check is needed.}
\end{rpcerrors}


% vx.rename
\section{vx.rename}

This method is used to rename a virtual server in the configuration database.
This does not involve filesystem changes, in particular, the virtual servers
root filesystem is not moved or renamed.

\begin{rpcsynopsis}{vx.rename}{string name, string newname}
\rpcparam{name}{Unique virtual server name}
\rpcparam{newname}{New unique virtual server name}
\end{rpcsynopsis}

\begin{rpcaccess}
\rpccapability{CREATE} and \rpcownerchecks.
\end{rpcaccess}

\rpcreturnnil

\begin{rpcerrors}
% FIXME: MERUNNING only needed in vx.move?
\rpcerror{MERUNNING}{The specified virtual server is currently running,
	therefore it cannot be renamed}
\rpcerror{MEEXIST}{The specified new virtual server name currently exists and
	cannot be used}
\end{rpcerrors}


% vx.restart
\section{vx.restart}

This method is used to restart virtual servers. It has the same effect as
\texttt{vx.reboot} except that it waits for the virtual server to stop and
startup again, which cannot be acomplished by the former. Technically, this is
a combination of \texttt{vx.stop} and \texttt{vx.start} only.

\begin{rpcsynopsis}{vx.restart}{string name}
\rpcparam{name}{Unique virtual server name}
\end{rpcsynopsis}

\begin{rpcaccess}
\rpccapability{INIT} and \rpcownerchecks.
\end{rpcaccess}

\rpcreturnnil

\begin{rpcerrors}
\rpcerror{MESTOPPED}{The specified virtual server is currently not running,
	therefore it cannot be stopped}
\end{rpcerrors}


% vx.start
\section{vx.start}

This method is used to start a virtual server.

\begin{rpcsynopsis}{vx.start}{string name}
\rpcparam{name}{Unique virtual server name}
\end{rpcsynopsis}

\begin{rpcaccess}
\rpccapability{INIT} and \rpcownerchecks.
\end{rpcaccess}

\rpcreturnnil

\begin{rpcerrors}
\rpcerror{MERUNNING}{The specified virtual server is currently running,
	therefore it cannot be started again}
\end{rpcerrors}


% vx.status
\section{vx.status}

This method is used to obtain status information about a virtual server.

\begin{rpcsynopsis}{vx.status}{string name}
\rpcparam{name}{Unique virtual server name}
\end{rpcsynopsis}

\begin{rpcaccess}
\rpccapability{INIT} and \rpcownerchecks.
\end{rpcaccess}

\begin{rpcreturncomplex}{\texttt{struct}}{bool running, int nproc, int uptime}
\rpcreturnparam{running}{This value is set to true if the specified virtual
	server is currently running, false otherwise}
\rpcreturnparam{nproc}{This value is set to the number of processes inside the
	specified virtual server, if it is running, 0 otherwise}
\rpcreturnparam{uptime}{This value is set to the uptime in seconds of the
	specified virtual server, if it is running, 0 otherwise}
\end{rpcreturncomplex}

\rpcnoerrors


% vx.stop
\section{vx.stop}

This method is used as a wrapper to vx.exec using the configured \texttt{halt}
command.

\begin{rpcsynopsis}{vx.stop}{string name}
\rpcparam{name}{Unique virtual server name}
\end{rpcsynopsis}

\begin{rpcaccess}
\rpccapability{INIT} and \rpcownerchecks.
\end{rpcaccess}

\rpcreturnsimple{the return value of \texttt{vx.exec}}

\begin{rpcerrors}
\rpcerror{MESTOPPED}{The specified virtual server is currently not running,
	therefore it cannot be stopped}
\end{rpcerrors}
