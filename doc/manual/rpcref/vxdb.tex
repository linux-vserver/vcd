\chapter{Database Manipulation Methods}

The \texttt{vxdb} family of functions provide manipulation facilities for the
configuration database.


% vxdb.dx.limit.get
\section{vxdb.dx.limit.get}

This method is used to get information about configured disk limits.

\begin{rpcsynopsis}{vxdb.dx.limit.get}{string name}
\rpcparam{name}{Unique virtual server name}
\end{rpcsynopsis}

\begin{rpcaccess}
\rpccapability{DLIM} and \rpcownerchecks.
\end{rpcaccess}

\begin{rpcreturncomplex}{a \texttt{struct}}{uint32 space, uint32 inodes,
	int reserved}
\rpcreturnparam{space}{Maximum amount of disk space for this virtual server
	in KB}
\rpcreturnparam{inodes}{Maximum number of inodes for this virtual server}
\rpcreturnparam{reserved}{Disk space reserved for the super-user (root)
	\emph{inside} the virtual server in percent}
\end{rpcreturncomplex}

\rpcnoerrors


% vxdb.dx.limit.remove
\section{vxdb.dx.limit.remove}

This method is used to remove information about configured disk limits.

\begin{rpcsynopsis}{vxdb.dx.limit.remove}{string name}
\rpcparam{name}{Unique virtual server name}
\end{rpcsynopsis}

\begin{rpcaccess}
\rpccapability{DLIM} and \rpcownerchecks.
\end{rpcaccess}

\rpcreturnnil

\rpcnoerrors


% vxdb.dx.limit.set
\section{vxdb.dx.limit.set}

This method is used to set (add \& change) information about configured disk
limits.

\begin{rpcsynopsis}{vxdb.dx.limit.set}{string name, uint32 space,
	uint32 inodes, int reserved}
\rpcparam{name}{Unique virtual server name}
\rpcparam{space}{Maximum amount of disk space for this virtual server in KB}
\rpcparam{inodes}{Maximum number of inodes for this virtual server}
\rpcparam{reserved}{Disk space reserved for the super-user (root)
	\emph{inside} the virtual server in percent}
\end{rpcsynopsis}

\begin{rpcaccess}
\rpccapability{DLIM} and \rpcownerchecks.
\end{rpcaccess}

\rpcreturnnil

\rpcnoerrors


% vxdb.init.get
\section{vxdb.init.get}

This method is used to get information about configured \texttt{init},
\texttt{halt} and \texttt{reboot} commands.

\begin{rpcsynopsis}{vxdb.init.get}{string name}
\rpcparam{name}{Unique virtual server name}
\end{rpcsynopsis}

\begin{rpcaccess}
\rpccapability{INIT} and \rpcownerchecks.
\end{rpcaccess}

\begin{rpcreturncomplex}{a \texttt{struct}}{string init, string halt,
	string reboot}
\rpcreturnparam{init}{Absolute path of the init command \emph{inside} the
	virtual server (defaults to \texttt{/sbin/init} if empty)}
\rpcreturnparam{halt}{Absolute path of the halt command \emph{inside} the
	virtual server (defaults to \texttt{/sbin/halt} if empty)}
\rpcreturnparam{reboot}{Absolute path of the reboot command \emph{inside} the
	virtual server (defaults to \texttt{/sbin/reboot} if empty)}
\end{rpcreturncomplex}

\rpcnoerrors


% vxdb.init.set
\section{vxdb.init.set}

This method is used to set (change) information about configured \texttt{init},
\texttt{halt} and \texttt{reboot} commands.

\begin{rpcsynopsis}{vxdb.init.set}{string name, string init, string halt,
	string reboot}
\rpcparam{name}{Unique virtual server name}
\rpcparam{init}{Absolute path of the init command \emph{inside} the virtual
	server (defaults to \texttt{/sbin/init} if empty)}
\rpcparam{halt}{Absolute path of the halt command \emph{inside} the virtual
	server (defaults to \texttt{/sbin/halt} if empty)}
\rpcparam{reboot}{Absolute path of the reboot command \emph{inside} the
	virtual server (defaults to \texttt{/sbin/reboot} if empty)}
\end{rpcsynopsis}

\rpcreturnnil

\rpcnoerrors


% vxdb.list
\section{vxdb.list}

This method is used to get a list of all currently configured and owned virtual
servers.

\begin{rpcsynopsis}{vxdb.list}{string username}
\rpcparam{username}{Unique username (this value may be empty to return a list
containing all virtual servers)}
\end{rpcsynopsis}

\begin{rpcaccess}
\rpcnocapability and \rpcownerchecks. The owner checks do not check the usual
\texttt{name} parameter here, instead it returns all virtual servers that would
pass owner checks.
\end{rpcaccess}


\rpcreturnsimple{an \texttt{array} of \texttt{string}s -- one for each virtual
	server --}

\rpcnoerrors


% vxdb.mount.get
\section{vxdb.mount.get}

This method is used to get information about configured mount points.

\begin{rpcsynopsis}{vxdb.mount.get}{string name, string dst}
\rpcparam{name}{Unique virtual server name}
\rpcparam{dst}{Absolute destination path for the mount point inside the
	virtual server (this value may be empty to return a list of all
	configured mount points)}
\end{rpcsynopsis}

\begin{rpcaccess}
\rpccapability{MOUNT} and \rpcownerchecks.
\end{rpcaccess}

\begin{rpcreturncomplex}{an \texttt{array} of \texttt{struct}s -- one for each
	mount point --}{string src, string dst, string type, string opts}
\rpcreturnparam{src}{Absolute source path \emph{inside} the virtual server or
	arbitrary identifier (defaults to \texttt{none} if empty)}
\rpcreturnparam{dst}{Absolute destination path for the mount point inside the
	virtual server}
\rpcreturnparam{type}{Filesystem type for the specified source path
	(defaults to \texttt{auto}} if empty)
\rpcreturnparam{opts}{Mount options for the specified filesystem
	(defaults to \texttt{defaults} if empty)}
\end{rpcreturncomplex}

\rpcnoerrors


% vxdb.mount.remove
\section{vxdb.mount.remove}

This method is used to remove information about configured mount points.

\begin{rpcsynopsis}{vxdb.mount.remove}{string name, string dst}
\rpcparam{name}{Unique virtual server name}
\rpcparam{dst}{Absolute destination path for the mount point inside the
	virtual server (this value may be empty to remove all configured mount
	points)}
\end{rpcsynopsis}

\begin{rpcaccess}
\rpccapability{MOUNT} and \rpcownerchecks.
\end{rpcaccess}

\rpcreturnnil

\rpcnoerrors


% vxdb.mount.set
\section{vxdb.mount.set}

This method is used to set (add \& change) information about configured mount
points.

\begin{rpcsynopsis}{vxdb.mount.set}{string name, string src, string dst,
	string type, string opts}
\rpcparam{name}{Unique virtual server name}
\rpcparam{src}{Absolute source path \emph{inside} the virtual server or
	arbitrary identifier (defaults to \texttt{none} if empty)}
\rpcparam{dst}{Absolute destination path for the mount point inside the
	virtual server}
\rpcparam{type}{Filesystem type for the specified source path (defaults to
	\texttt{auto} if empty)}
\rpcparam{opts}{Mount options for the specified filesystem (defaults to
	\texttt{defaults} if empty)}
\end{rpcsynopsis}

\begin{rpcaccess}
\rpccapability{MOUNT} and \rpcownerchecks.
\end{rpcaccess}

\rpcreturnnil

\rpcnoerrors


% vxdb.name.get
\section{vxdb.name.get}

This method is used to lookup the name of a virtual server by its corresponding
context ID.

\begin{rpcsynopsis}{vxdb.name.get}{int xid}
\rpcparam{xid}{Unique context ID}
\end{rpcsynopsis}

\begin{rpcaccess}
\rpccapability{INFO} and \rpcnoownerchecks.
\end{rpcaccess}

\rpcreturnsimple{a \texttt{string} containing the virtual server name}

\rpcnoerrors


% vxdb.nx.addr.get
\section{vxdb.nx.addr.get}

This method is used to get information about configured network addresses.

\begin{rpcsynopsis}{vxdb.nx.addr.get}{string name, string addr}
\rpcparam{name}{Unique virtual server name}
\rpcparam{addr}{Network address in dot-decimal form (this value may be empty
	to return a list of all configured network adresses)}
\end{rpcsynopsis}

\begin{rpcaccess}
\rpccapability{NET} and \rpcownerchecks.
\end{rpcaccess}

\begin{rpcreturncomplex}{an \texttt{array} of \texttt{struct}s -- one for each
	network address --}{string addr, string netmask}
\rpcreturnparam{addr}{Network address in dot-decimal form}
\rpcreturnparam{netmask}{Network mask in dot-decimal form (defaults to
	\texttt{255.255.255.0} if empty)}
\end{rpcreturncomplex}

\rpcnoerrors


% vxdb.nx.addr.remove
\section{vxdb.nx.addr.remove}

This method is used to remove information about configured network addresses.

\begin{rpcsynopsis}{vxdb.nx.addr.remove}{string name, string addr}
\rpcparam{name}{Unique virtual server name}
\rpcparam{addr}{Network address in dot-decimal form (this value may be empty to
	remove all configured network adresses)}
\end{rpcsynopsis}

\begin{rpcaccess}
\rpccapability{NET} and \rpcownerchecks.
\end{rpcaccess}

\rpcreturnnil

\rpcnoerrors


% vxdb.nx.addr.set
\section{vxdb.nx.addr.set}

This method is used to set (add \& change) information about configured network
addresses.

\begin{rpcsynopsis}{vxdb.nx.addr.set}{string name, string addr, string netmask}
\rpcparam{name}{Unique virtual server name}
\rpcparam{addr}{Network address in dot-decimal form}
\rpcparam{netmask}{Network mask in dot-decimal form (defaults to
	\texttt{255.255.255.0} if empty)}
\end{rpcsynopsis}

\begin{rpcaccess}
\rpccapability{NET} and \rpcownerchecks.
\end{rpcaccess}

\rpcreturnnil

\rpcnoerrors


% vxdb.nx.broadcast.get
\section{vxdb.nx.broadcast.get}

This method is used to get information about the configured broadcast address.

\begin{rpcsynopsis}{vxdb.nx.broadcast.get}{string name}
\rpcparam{name}{Unique virtual server name}
\end{rpcsynopsis}

\begin{rpcaccess}
\rpccapability{NET} and \rpcownerchecks.
\end{rpcaccess}

\rpcreturnsimple{a \texttt{string} containing the broadcast address}

\rpcnoerrors


% vxdb.nx.broadcast.remove
\section{vxdb.nx.broadcast.remove}

This method is used to remove information about the configured broadcast
address.

\begin{rpcsynopsis}{vxdb.nx.broadcast.remove}{string name}
\rpcparam{name}{Unique virtual server name}
\end{rpcsynopsis}

\begin{rpcaccess}
\rpccapability{NET} and \rpcownerchecks.
\end{rpcaccess}

\rpcreturnnil

\rpcnoerrors


% vxdb.vx.broadcast.set
\section{vxdb.nx.broadcast.set}

This method is used to set (change) information about the configured broadcast
address.

\begin{rpcsynopsis}{vxdb.nx.broadcast.set}{string name, string broadcast}
\rpcparam{name}{Unique virtual server name}
\rpcparam{broadcast}{Broadcast address in dot-decimal form}
\end{rpcsynopsis}

\begin{rpcaccess}
\rpccapability{NET} and \rpcownerchecks.
\end{rpcaccess}

\rpcreturnnil

\rpcnoerrors


% vxdb.owner.add
\section{vxdb.owner.add}

This method is used to add users to the list of configured owners in the
internal user database.

\begin{rpcsynopsis}{vxdb.owner.add}{string name, string username}
\rpcparam{name}{Unique virtual server name}
\rpcparam{username}{Unique username}
\end{rpcsynopsis}

\begin{rpcaccess}
\rpccapability{AUTH} and \rpcnoownerchecks.
\end{rpcaccess}

\rpcreturnnil

\rpcnoerrors


% vxdb.owner.get
\section{vxdb.owner.get}

This method is used to get information about configured owners in the internal
user database.

\begin{rpcsynopsis}{vxdb.owner.get}{string name}
\rpcparam{name}{Unique virtual server name}
\end{rpcsynopsis}

\begin{rpcaccess}
\rpccapability{AUTH} and \rpcnoownerchecks.
\end{rpcaccess}

\rpcreturnsimple{an \texttt{array} of \texttt{string}s -- one for each
	owner --}

\rpcnoerrors


% vxdb.owner.remove
\section{vxdb.owner.remove}

This method is used to remove information about configured owners from the
internal user database.

\begin{rpcsynopsis}{vxdb.owner.remove}{string name, string username}
\rpcparam{name}{Unique virtual server name}
\rpcparam{username}{Unique username (this value may be empty to remove all
	configured owners)}
\end{rpcsynopsis}

\begin{rpcaccess}
\rpccapability{AUTH} and \rpcnoownerchecks.
\end{rpcaccess}

\rpcreturnnil

\rpcnoerrors


% vxdb.vdir.get
\section{vxdb.vdir.get}

This method is used to get the absolute root filesystem path of a virtual
server.

\begin{rpcsynopsis}{vxdb.vdir.get}{string name}
\rpcparam{name}{Unique virtual server name}
\end{rpcsynopsis}

\begin{rpcaccess}
\rpccapability{INFO} and \rpcnoownerchecks.
\end{rpcaccess}

\rpcreturnsimple{a \texttt{string} containing the absolute root filesystem
	path}

\rpcnoerrors


% vxdb.vx.bcaps.add
\section{vxdb.vx.bcaps.add}

This method is used to add information about configured system capabilities.

\begin{rpcsynopsis}{vxdb.vx.bcaps.add}{string name, string bcap}
\rpcparam{name}{Unique virtual server name}
\rpcparam{bcap}{System capability to add}
\end{rpcsynopsis}

\begin{rpcaccess}
\rpccapability{BCAP} and \rpcownerchecks.
\end{rpcaccess}

\rpcreturnnil

\rpcnoerrors


% vxdb.vx.bcaps.get
\section{vxdb.vx.bcaps.get}

This method is used to get information about configured system capabilities.

\begin{rpcsynopsis}{vxdb.vx.bcaps.get}{string name}
\rpcparam{name}{Unique virtual server name}
\end{rpcsynopsis}

\begin{rpcaccess}
\rpccapability{BCAP} and \rpcownerchecks.
\end{rpcaccess}

\rpcreturnsimple{an \texttt{array} of \texttt{string}s -- one for each system
	capability --}

\rpcnoerrors


% vxdb.vx.bcaps.remove
\section{vxdb.vx.bcaps.remove}

This method is used to remove information about configured system capabilities.

\begin{rpcsynopsis}{vxdb.vx.bcaps.remove}{string name, string bcap}
\rpcparam{name}{Unique virtual server name}
\rpcparam{bcap}{System capability to remove}
\end{rpcsynopsis}

\begin{rpcaccess}
\rpccapability{BCAP} and \rpcownerchecks.
\end{rpcaccess}

\rpcreturnnil

\rpcnoerrors


% vxdb.vx.ccaps.add
\section{vxdb.vx.ccaps.add}

This method is used to add information about configured context capabilities.

\begin{rpcsynopsis}{vxdb.vx.ccaps.add}{string name, string ccap}
\rpcparam{name}{Unique virtual server name}
\rpcparam{ccap}{Context capability to add}
\end{rpcsynopsis}

\begin{rpcaccess}
\rpccapability{CCAP} and \rpcownerchecks.
\end{rpcaccess}

\rpcreturnnil

\rpcnoerrors


% vxdb.vx.ccaps.get
\section{vxdb.vx.ccaps.get}

This method is used to get information about configured context capabilities.

\begin{rpcsynopsis}{vxdb.vx.ccaps.get}{string name}
\rpcparam{name}{Unique virtual server name}
\end{rpcsynopsis}

\begin{rpcaccess}
\rpccapability{CCAP} and \rpcownerchecks.
\end{rpcaccess}

\rpcreturnsimple{an \texttt{array} of \texttt{string}s -- one for each context
	capability --}

\rpcnoerrors


% vxdb.vx.ccaps.remove
\section{vxdb.vx.ccaps.remove}

This method is used to remove information about configured context
capabilities.

\begin{rpcsynopsis}{vxdb.vx.ccaps.remove}{string name, string ccap}
\rpcparam{name}{Unique virtual server name}
\rpcparam{ccap}{Context capability to remove}
\end{rpcsynopsis}

\begin{rpcaccess}
\rpccapability{CCAP} and \rpcownerchecks.
\end{rpcaccess}

\rpcreturnnil

\rpcnoerrors


% vxdb.vx.flags.add
\section{vxdb.vx.flags.add}

This method is used to add information about configured context flags.

\begin{rpcsynopsis}{vxdb.vx.flags.add}{string name, string flag}
\rpcparam{name}{Unique virtual server name}
\rpcparam{flag}{Context flag to add}
\end{rpcsynopsis}

\begin{rpcaccess}
\rpccapability{CFLAG} and \rpcownerchecks.
\end{rpcaccess}

\rpcreturnnil

\rpcnoerrors


% vxdb.vx.flags.get
\section{vxdb.vx.flags.get}

This method is used to get information about configured context flags.

\begin{rpcsynopsis}{vxdb.vx.flags.get}{string name}
\rpcparam{name}{Unique virtual server name}
\end{rpcsynopsis}

\begin{rpcaccess}
\rpccapability{CFLAG} and \rpcownerchecks.
\end{rpcaccess}

\rpcreturnsimple{an \texttt{array} of \texttt{string}s -- one for each context
	flag -- }

\rpcnoerrors


% vxdb.vx.flags.remove
\section{vxdb.vx.flags.remove}

This method is used to remove information about configured context flags.

\begin{rpcsynopsis}{vxdb.vx.flags.remove}{string name, string flag}
\rpcparam{name}{Unique virtual server name}
\rpcparam{flag}{Context flag to remove}
\end{rpcsynopsis}

\begin{rpcaccess}
\rpccapability{CFLAG} and \rpcownerchecks.
\end{rpcaccess}

\rpcreturnnil

\rpcnoerrors


% vxdb.vx.limit.get
\section{vxdb.vx.limit.get}

This method is used to get information about configured resource limits.

\begin{rpcsynopsis}{vxdb.vx.limit.get}{string name, string limit}
\rpcparam{name}{Unique virtual server name}
\rpcparam{limit}{Limit type to get (this value may be empty to retrieve all
	configured resource limits)}
\end{rpcsynopsis}

\begin{rpcaccess}
\rpccapability{RLIM} and \rpcownerchecks.
\end{rpcaccess}

\begin{rpcreturncomplex}{an \texttt{array} of \texttt{struct}s -- one for each
	resource limit --}{string limit, uint64 soft, uint64 max}
\rpcreturnparam{limit}{Limit type to add or change}
\rpcreturnparam{soft}{Softl imit for the specified type}
\rpcreturnparam{max}{Hard limit for the specified type}
\end{rpcreturncomplex}

\rpcnoerrors


% vxdb.vx.limit.remove
\section{vxdb.vx.limit.remove}

This method is used to remove information about configured resource limits.

\begin{rpcsynopsis}{vxdb.vx.limit.remove}{string name, string limit}
\rpcparam{name}{Unique virtual server name}
\rpcparam{limit}{Limit type to remove (this value may be empty to remove all
	configured resource limits)}
\end{rpcsynopsis}

\begin{rpcaccess}
\rpccapability{RLIM} and \rpcownerchecks.
\end{rpcaccess}

\rpcreturnnil

\rpcnoerrors


% vxdb.vx.limit.set
\section{vxdb.vx.limit.set}

This method is used to set (add \& change) information about configured
resource limits.

\begin{rpcsynopsis}{vxdb.vx.limit.set}{string name, string limit, uint64 soft,
	uint64 max}
\rpcparam{name}{Unique virtual server name}
\rpcparam{limit}{Limit type to add or change}
\rpcparam{soft}{Softl imit for the specified type}
\rpcparam{max}{Hard limit for the specified type}
\end{rpcsynopsis}

\begin{rpcaccess}
\rpccapability{RLIM} and \rpcownerchecks.
\end{rpcaccess}

\rpcreturnnil

\rpcnoerrors


% vxdb.vx.sched.get
\section{vxdb.vx.sched.get}

This method is used to get information about configured CPU scheduler buckets.

\begin{rpcsynopsis}{vxdb.vx.sched.get}{string name, int cpuid}
\rpcparam{name}{Unique virtual server name}
\rpcparam{cpuid}{CPU ID as listed in \texttt{/proc/cpuinfo}}
\end{rpcsynopsis}

\begin{rpcaccess}
\rpccapability{SCHED} and \rpcownerchecks.
\end{rpcaccess}

\begin{rpcreturncomplex}{an \texttt{array} of \texttt{struct}s -- one for each
	CPU ID --}{int cpuid, int interval, int fillrate, int interval2,
	int fillrate2, int tokensmin, int tokensmax}
\rpcreturnparam{cpuid}{CPU ID as listed in \texttt{/proc/cpuinfo}}
\rpcreturnparam{interval}{Interval between fills in jiffies}
\rpcreturnparam{fillrate}{Tokens to fill each interval}
\rpcreturnparam{interval2}{Interval between fills in jiffies (IDLE time
	setting)}
\rpcreturnparam{fillrate2}{Tokens to fill each interval (IDLE time setting)}
\rpcreturnparam{tokensmin}{Minimum number of tokens to schedule processes}
\rpcreturnparam{tokensmax}{Maximum number of tokens in the bucket}
\end{rpcreturncomplex}

\rpcnoerrors


% vxdb.vx.sched.remove
\section{vxdb.vx.sched.remove}

This method is used to remove information about configured CPU scheduler
buckets.

\begin{rpcsynopsis}{vxdb.vx.sched.remove}{string name, int cpuid}
\rpcparam{name}{Unique virtual server name}
\rpcparam{cpuid}{CPU ID as listed in \texttt{/proc/cpuinfo}}
\end{rpcsynopsis}

\begin{rpcaccess}
\rpccapability{SCHED} and \rpcownerchecks.
\end{rpcaccess}

\rpcreturnnil

\rpcnoerrors


% vxdb.vx.sched.set
\section{vxdb.vx.sched.set}

This method is used to set (add \& change) information about configured CPU
scheduler buckets.

\begin{rpcsynopsis}{vxdb.vx.sched.set}{string name, int cpuid, int interval,
	int fillrate, int interval2, int fillrate2, int tokensmin, int tokensmax}
\rpcparam{name}{Unique virtual server name}
\rpcparam{cpuid}{CPU ID as listed in \texttt{/proc/cpuinfo}}
\rpcparam{interval}{Interval between fills in jiffies}
\rpcparam{fillrate}{Tokens to fill each interval}
\rpcparam{interval2}{Interval between fills in jiffies (IDLE time setting)}
\rpcparam{fillrate2}{Tokens to fill each interval (IDLE time setting)}
\rpcparam{tokensmin}{Minimum number of tokens to schedule processes}
\rpcparam{tokensmax}{Maximum number of tokens in the bucket}
\end{rpcsynopsis}

\begin{rpcaccess}
\rpccapability{SCHED} and \rpcownerchecks.
\end{rpcaccess}

\rpcreturnnil

\rpcnoerrors


% vxdb.vx.uname.get
\section{vxdb.vx.uname.get}

This method is used to get information about configured virtual system
information.

\begin{rpcsynopsis}{vxdb.vx.uname.get}{string name, string uname}
\rpcparam{name}{Unique virtual server name}
\rpcparam{uname}{System information type (this value may be empty to retrieve
	information about all configured system information types)}
\end{rpcsynopsis}

\begin{rpcaccess}
\rpccapability{UNAME} and \rpcownerchecks.
\end{rpcaccess}

\begin{rpcreturncomplex}{an \texttt{array} of \texttt{struct}s -- one for each
	system information type --}{string uname, string value}
\rpcreturnparam{uname}{System information type}
\rpcreturnparam{value}{System information value}
\end{rpcreturncomplex}

\rpcnoerrors


% vxdb.vx.uname.remove
\section{vxdb.vx.uname.remove}

This method is used to remove information about configured virtual system
information.

\begin{rpcsynopsis}{vxdb.vx.uname.remove}{string name, string uname}
\rpcparam{name}{Unique virtual server name}
\rpcparam{uname}{System information type}
\end{rpcsynopsis}

\begin{rpcaccess}
\rpccapability{UNAME} and \rpcownerchecks.
\end{rpcaccess}

\rpcreturnnil

\rpcnoerrors


% vxdb.vx.uname.set
\section{vxdb.vx.uname.set}

This method is used to set (add \& change) information about configured virtual
system information.

\begin{rpcsynopsis}{vxdb.vx.uname.set}{string name, string uname, string value}
\rpcparam{name}{Unique virtual server name}
\rpcparam{uname}{System information type}
\rpcparam{value}{System information value to set}
\end{rpcsynopsis}

\begin{rpcaccess}
\rpccapability{UNAME} and \rpcownerchecks.
\end{rpcaccess}

\rpcreturnnil

\rpcnoerrors


% vxdb.xid.get
\section{vxdb.xid.get}

This method is used to lookup the Context ID of a virtual server by its name.

\begin{rpcsynopsis}{vxdb.xid.get}{string name}
\rpcparam{name}{Unique virtual server name}
\end{rpcsynopsis}

\begin{rpcaccess}
\rpccapability{INFO} and \rpcnoownerchecks.
\end{rpcaccess}

\rpcreturnsimple{an \texttt{int} containing the Context ID of the specified
	virtual server}

\rpcnoerrors
