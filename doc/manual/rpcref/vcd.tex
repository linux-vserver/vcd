\chapter{Daemon Administration Methods}


% vcd.login
\section{vcd.login}

This method is used to authenticate against the user database. Internally it is
a no-op method, since every method call must be authenticated. This is only
usefull for GUIs or web panels, to let users login without actually doing
something.

\rpcsynopsisempty{vcd.login}

\begin{rpcaccess}
\rpcnocapability and \rpcnoownerchecks.
\end{rpcaccess}

\rpcreturnnil

\rpcnoerrors


% vcd.status
\section{vcd.status}

This method is used to retreive internal daemon statistics. The daemon stores a
lot of runtime information in VXDB that can be used by system administrators or
user to check the daemon or account healthiness respectively.

\rpcsynopsisempty{vcd.status}

\begin{rpcaccess}
\rpccapability{INFO} and \rpcnoownerchecks.
\end{rpcaccess}

\rpcreturnnil

\rpcnoerrors



% vcd.user.caps.add
\section{vcd.user.caps.add}

This method is used to add capabilities to the internal user database.

\begin{rpcsynopsis}{vcd.user.caps.add}{string username, string cap}
\rpcparam{username}{Unique username}
\rpcparam{cap}{Capability to add}
\end{rpcsynopsis}

\begin{rpcaccess}
\rpccapability{AUTH} and \rpcnoownerchecks.
\end{rpcaccess}

\rpcreturnnil

\rpcnoerrors


% vcd.user.caps.get
\section{vcd.user.caps.get}

This method is used to get information about configured capabilities in the
internal user database.

\begin{rpcsynopsis}{vcd.user.caps.get}{string username}
\rpcparam{username}{Unique username}
\end{rpcsynopsis}

\begin{rpcaccess}
\rpccapability{AUTH} and \rpcnoownerchecks.
\end{rpcaccess}

\rpcreturnsimple{an \texttt{array} of \texttt{string}s - on for each
	capability - }

\rpcnoerrors


% vcd.user.caps.remove
\section{vcd.user.caps.remove}

This method is used to remove information about configured capabilities from
the internal user database.

\begin{rpcsynopsis}{vcd.user.caps.remove}{string username, string cap}
\rpcparam{username}{Unique username}
\rpcparam{cap}{Capability to remove}
\end{rpcsynopsis}

\begin{rpcaccess}
\rpccapability{AUTH} and \rpcnoownerchecks.
\end{rpcaccess}

\rpcreturnnil

\rpcnoerrors


% vcd.user.get
\section{vcd.user.get}

This method is used to get information about configured users in the internal
user database.

\begin{rpcsynopsis}{vcd.user.get}{string username}
\rpcparam{username}{Unique username}
\end{rpcsynopsis}

\begin{rpcaccess}
\rpccapability{AUTH} and \rpcnoownerchecks.
\end{rpcaccess}

\begin{rpcreturncomplex}{\texttt{struct}}{string password, bool admin}
\rpcreturnparam{password}{Password hash for the specified user}
\rpcreturnparam{admin}{Set the administrator flag for the specified user}
\end{rpcreturncomplex}


% vcd.user.remove
\section{vcd.user.remove}

This method is used to remove information about configured users from the
internal user database.

\begin{rpcsynopsis}{vcd.user.remove}{string username}
\rpcparam{username}{Unique username}
\end{rpcsynopsis}

\begin{rpcaccess}
\rpccapability{AUTH} and \rpcnoownerchecks.
\end{rpcaccess}

\rpcreturnnil

\rpcnoerrors


% vcd.user.set
\section{vcd.user.set}

This method is used to set (add \& change) information about configured users
in the internal user database.

\begin{rpcsynopsis}{vcd.user.set}{string username, string password,
	bool admin}
\rpcparam{username}{Unique username}
\rpcparam{password}{Password for the specified user (if user already exists
	this value may be empty to not change the password)}
\rpcparam{admin}{Set the administrator flag for the specified user}
\end{rpcsynopsis}

\begin{rpcaccess}
\rpccapability{AUTH} and \rpcnoownerchecks.
\end{rpcaccess}

\rpcreturnnil

\rpcnoerrors


